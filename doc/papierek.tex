%%%%%%%%%%%%%%%%%%%%%%%%%%%%%%%%%%%%%%%%%
% Thin Sectioned Essay
% LaTeX Template
% Version 1.0 (3/8/13)
%
% This template has been downloaded from:
% http://www.LaTeXTemplates.com
%
% Original Author:
% Nicolas Diaz (nsdiaz@uc.cl) with extensive modifications by:
% Vel (vel@latextemplates.com)
%
% License:
% CC BY-NC-SA 3.0 (http://creativecommons.org/licenses/by-nc-sa/3.0/)
%
%%%%%%%%%%%%%%%%%%%%%%%%%%%%%%%%%%%%%%%%%

%----------------------------------------------------------------------------------------
%	PACKAGES AND OTHER DOCUMENT CONFIGURATIONS
%----------------------------------------------------------------------------------------

\documentclass[a4paper, 11pt]{article} % Font size (can be 10pt, 11pt or 12pt) and paper size (remove a4paper for US letter paper)

\usepackage[protrusion=true,expansion=true]{microtype} % Better typography
\usepackage{graphicx} % Required for including pictures
\usepackage{wrapfig} % Allows in-line images
\usepackage{listings}             % Include the listings-package
\usepackage[utf8]{inputenc}
\usepackage{mathpazo} % Use the Palatino font
\usepackage[T1]{fontenc} % Required for accented characters
\linespread{1.05} % Change line spacing here, Palatino benefits from a slight increase by default
\usepackage{color}
\makeatletter
\renewcommand\@biblabel[1]{\textbf{#1.}} % Change the square brackets for each bibliography item from '[1]' to '1.'
\renewcommand{\@listI}{\itemsep=0pt} % Reduce the space between items in the itemize and enumerate environments and the bibliography

\lstset{numbers=left, frame=single,}

\definecolor{codegreen}{rgb}{0,0.6,0}
\definecolor{codegray}{rgb}{0.5,0.5,0.5}
\definecolor{codepurple}{rgb}{0.58,0,0.82}
\definecolor{backcolour}{rgb}{0.95,0.95,0.92}
 
\lstdefinestyle{mystyle}{
    backgroundcolor=\color{backcolour},   
    commentstyle=\color{codegreen},
    keywordstyle=\color{magenta},
    numberstyle=\tiny\color{codegray},
    stringstyle=\color{codepurple},
    basicstyle=\footnotesize,
    breakatwhitespace=false,         
    breaklines=true,                 
    captionpos=b,                    
    keepspaces=true,                 
    numbers=left,                    
    numbersep=5pt,                  
    showspaces=false,                
    showstringspaces=false,
    showtabs=false,                  
    tabsize=2
}

\renewcommand{\maketitle}{ % Customize the title - do not edit title and author name here, see the TITLE block below
\begin{flushright} % Right align
{\LARGE\@title} % Increase the font size of the title

\vspace{50pt} % Some vertical space between the title and author name

{\large\@author} % Author name
\\\@date % Date

\vspace{40pt} % Some vertical space between the author block and abstract
\end{flushright}
}

%----------------------------------------------------------------------------------------
%	TITLE
%----------------------------------------------------------------------------------------

\title{\textbf{Przegląd wybranych generatorów liczb pseudolosowych}\\ % Title
i ich analiza pod kątem losowości.} % Subtitle

\author{\textsc{Mateusz Sołtysik, Andrzej Kwak, Wiktor Dyngosz} % Author
\\{Politechnika Wrocławska, \textit{Wydział Podstawowych Problemów Techniki}}} % Institution

\date{\today} % Date

%----------------------------------------------------------------------------------------

\begin{document}

\maketitle % Print the title section

%----------------------------------------------------------------------------------------
%	ABSTRACT AND KEYWORDS
%----------------------------------------------------------------------------------------

%\renewcommand{\abstractname}{Summary} % Uncomment to change the name of the abstract to something else

\begin{abstract}
\end{abstract}

\hspace*{3,6mm}\textit{Keywords:} 

\vspace{30pt} % Some vertical space between the abstract and first section

\section*{Wstęp}

\section{Opis i analiza poszczególnych algorytmów}

% Mati tutaj pisze
\subsection{Blum Blum Shub}
Algorytm Blum Blum Shub został zaproponowany przez Lenore Blum, Manuela Blum oraz Michaela Shub'a w pracy pt. "A Simple Unpredictable Pseudo-Random Number Generator"\footnote{http://epubs.siam.org/doi/abs/10.1137/0215025}.
\begin{lstlisting}[style=mystyle, language=java, frame=single]
private static final int p = 11;
private static final int q = 19;

public static long getRandomNumber() {
    seed = (seed * seed) % (p * q);
    return Math.abs(seed);
}
\end{lstlisting}

\subsection{Linear congruential generator}
\begin{lstlisting}[style=mystyle, language=java, frame=single]
private final static long a = 25173;
private final static long b = 13849;
private final static long m = 32768;

public static long getRandomNumber() {
    seed = (a * seed + b) % m;
    return seed;
}
\end{lstlisting}

\subsection{Mersenne twister}


\subsection{Park–Miller random number generator}

\begin{lstlisting}[style=mystyle, language=java, frame=single]
private static final long max = ((long) 2 << 30) - 1;
private static final long a = 16807;

public static long getRandomNumber() {
    seed = (a * seed) % max;
    return seed;
}
\end{lstlisting}

\subsection{Xorshift}
Algorytm został stworony przez George Marsaglia\footnote{http://www.jstatsoft.org/v08/i14/paper}. Zasada działania opiera się na generowaniu następnych numerów wielokrotnie biorąc różnicę symetryczną z niego i przesuniętej bitowo wersji tej liczby.


\begin{lstlisting}[style=mystyle, language=java, frame=single]
public static long getRandomNumber() {
    seed ^= seed >> 12;
    seed ^= seed << 25;
    seed ^= seed >> 27;
    seed = (seed * 2685821657736338717L) % max;
    return Math.abs(seed);
}
\end{lstlisting}

\section*{Wnioski}

%----------------------------------------------------------------------------------------
%	BIBLIOGRAPHY
%----------------------------------------------------------------------------------------

\bibliographystyle{unsrt}

\bibliography{sample}

%----------------------------------------------------------------------------------------

\end{document}