%%%%%%%%%%%%%%%%%%%%%%%%%%%%%%%%%%%%%%%%%
% Thin Sectioned Essay
% LaTeX Template
% Version 1.0 (3/8/13)
%
% This template has been downloaded from:
% http://www.LaTeXTemplates.com
%
% Original Author:
% Nicolas Diaz (nsdiaz@uc.cl) with extensive modifications by:
% Vel (vel@latextemplates.com)
%
% License:
% CC BY-NC-SA 3.0 (http://creativecommons.org/licenses/by-nc-sa/3.0/)
%
%%%%%%%%%%%%%%%%%%%%%%%%%%%%%%%%%%%%%%%%%

%----------------------------------------------------------------------------------------
%	PACKAGES AND OTHER DOCUMENT CONFIGURATIONS
%----------------------------------------------------------------------------------------

\documentclass[a4paper, 11pt]{article} % Font size (can be 10pt, 11pt or 12pt) and paper size (remove a4paper for US letter paper)

\usepackage[protrusion=true,expansion=true]{microtype} % Better typography
\usepackage{graphicx} % Required for including pictures
\usepackage{wrapfig} % Allows in-line images
\usepackage{listings}             % Include the listings-package
\usepackage[utf8]{inputenc}
\usepackage{mathpazo} % Use the Palatino font
\usepackage[T1]{fontenc} % Required for accented characters
\linespread{1.05} % Change line spacing here, Palatino benefits from a slight increase by default
\usepackage{color}
\usepackage{graphicx}
\makeatletter
\renewcommand\@biblabel[1]{\textbf{#1.}} % Change the square brackets for each bibliography item from '[1]' to '1.'
\renewcommand{\@listI}{\itemsep=0pt} % Reduce the space between items in the itemize and enumerate environments and the bibliography

\lstset{numbers=left, frame=single,}

\definecolor{codegreen}{rgb}{0,0.6,0}
\definecolor{codegray}{rgb}{0.5,0.5,0.5}
\definecolor{codepurple}{rgb}{0.58,0,0.82}
\definecolor{backcolour}{rgb}{0.95,0.95,0.92}
 
\lstdefinestyle{mystyle}{
    backgroundcolor=\color{backcolour},   
    commentstyle=\color{codegreen},
    keywordstyle=\color{magenta},
    numberstyle=\tiny\color{codegray},
    stringstyle=\color{codepurple},
    basicstyle=\footnotesize,
    breakatwhitespace=false,         
    breaklines=true,                 
    captionpos=b,                    
    keepspaces=true,                 
    numbers=left,                    
    numbersep=5pt,                  
    showspaces=false,                
    showstringspaces=false,
    showtabs=false,                  
    tabsize=2
}

\renewcommand{\maketitle}{ % Customize the title - do not edit title and author name here, see the TITLE block below
\begin{flushright} % Right align
{\LARGE\@title} % Increase the font size of the title

\vspace{50pt} % Some vertical space between the title and author name

{\large\@author} % Author name
\\\@date % Date

\vspace{40pt} % Some vertical space between the author block and abstract
\end{flushright}
}

%----------------------------------------------------------------------------------------
%	TITLE
%----------------------------------------------------------------------------------------

\title{\textbf{Przegląd wybranych generatorów liczb pseudolosowych}\\ % Title
i analiza ich widma spektralnego.} % Subtitle

\author{\textsc{Mateusz Sołtysik, Andrzej Kwak, Wiktor Dyngosz} % Author
\\{Politechnika Wrocławska, \textit{Wydział Podstawowych Problemów Techniki}}} % Institution

\date{Czerwiec, 2014} % Date

%----------------------------------------------------------------------------------------

\begin{document}

\maketitle % Print the title section

%----------------------------------------------------------------------------------------
%	ABSTRACT AND KEYWORDS
%----------------------------------------------------------------------------------------

%\renewcommand{\abstractname}{Summary} % Uncomment to change the name of the abstract to something else

\section*{Wstęp}
Pseudo-Random Number Generator (PRNG) – program lub podprogram, który na podstawie niewielkiej ilości informacji (ziarno, ang. seed) generuje deterministyczny, potencjalnie nieskończony, ciąg liczb. Generatory liczb pseudolosowych nie generują ciągów prawdziwie losowych - generator inicjowany ziarnem, które może przyjąć $k$ różnych wartości, jest w stanie wyprodukować co najwyżej $k$ różnych ciągów liczb. Ponieważ rozmiar zmiennych reprezentujących wewnętrzny stan generatora jest ograniczony i może on znajdować się tylko w ograniczonej liczbie stanów, po pewnym czasie generator dokona pełnego cyklu i zacznie generować te same wartości. Teoretyczny limit długości cyklu wyrażony jest przez $2^n$, gdzie $n$ to liczba bitów przeznaczonych na przechowywanie stanu wewnętrznego. W praktyce, większość generatorów ma znacznie krótsze okresy.
\\Pożądane cechy generatorów:
\begin{enumerate}
\item trudne do ustalenia ziarno, choć znany jest ciąg wygenerowanych bitów
\item trudne do ustalenia kolejno generowane bity, choć znany jest ciąg bitów dotychczas wygenerowanych
\end{enumerate}
Przykłady zastosowań generatorów liczb pseudolosowych:
\begin{enumerate}
\item Kryptografia
\item Całkowanie numeryczne (metoda Monte-Carlo)
\item Symulacja systemów masowej obsługi
\item Automaty do gier losowych
\end{enumerate}
W tej pracy przedstawiamy opis, zasadę działania oraz analizę widma spektralnego wymienionych algorytmów:
\begin{enumerate}
\item Blum Blum Shub
\item Liniowy Generator Kongruentny (LCG)
\item Mersenne twister
\item Generator Parka-Millera
\item Xorshift
\end{enumerate}
todo
\pagebreak
%----------------------------------------------------------------------------------------
\section{Opis i analiza algorytmów PRNG}
\subsection*{Informacje wstępne}
Każdy - opisany w tej pracy - algorytm, powinien być inicjowany tzw. seedem. Przypomnijmy, że dany algorytm z ustalonymi parametrami i znaną wartością seed, generuje te same ciągi! Aby zapobiec przekłamaniom w tym zakresie postanowiliśmy, że nasza implementacja będzie korzystała z funkcji $getSeed()$, która przy procesie generacji ziarna wykorzystuje niedeterministyczne źródło bitów ($/dev/urandom$). \begin{lstlisting}[style=mystyle, language=java, frame=single, caption = Przykładowa funkcja generująca seed.]
public static long getSeed() {
      SecureRandom random = new SecureRandom();
      byte seed[] = random.generateSeed(20);
      return Longs.fromByteArray(seed);
}
\end{lstlisting}
\subsection*{Testy spektralne}
Analiza spektralna algorytmu służącego do generowania liczb pseudolosowych polega na wygenerowaniu liczby próbek $p$ par uporządkowanych kolejnych liczb $(x_{i}, x_{i+1})$ dla $i \in 0,1,2,\dots,p-1$. W ramach tej pracy została zaimplementowana lista algorytmów oraz stworzony program GUI, który umożliwia przeprowadzanie owych testów. \\
Testy dla każdego algorytmu nie powinny przekraczać wartości $10^{9}$, ponieważ zakres 32-bitowej liczby jest równy $\frac{32\log(2)}{\log(10)} \approx 9$.

%----------------------------------------------------------------------------------------
\subsection{Blum Blum Shub}
Algorytm \textit{Blum Blum Shub} został zaproponowany przez Lenore Blum, Manuela Blum'a oraz Michaela Shub'a wiosną, 1986. roku, w pracy: "A Simple Unpredictable Pseudo-Random Number Generator"\footnote{http://epubs.siam.org/doi/abs/10.1137/0215025}. Generator ten jest postaci:
\[ x_{n+1} = (x_{n})^2 \bmod N \]
gdzie $x_{n+1}$ to kolejny stan generatora, $N$ to iloczyn dwóch dużych liczb pierwszych $p$ i $q$ takich, że:
\begin{enumerate}
\item dają w dzieleniu przez 4 resztę 3 ($p\equiv q \equiv 3 \bmod 4$),
\item mają możliwie mały $NWD(\phi(p-1), \phi(q-1))$, co zapewnia długi cykl.
\end{enumerate}
Wynikiem generatora jest kilka ostatnich bitów $x_{n}$.
\begin{lstlisting}[style=mystyle, language=java, frame=single, caption = Generowanie następnej liczby pseudolosowej przez BBS]
private static final int p = 11;
private static final int q = 19;

public static long getRandomNumber() {
    seed = (seed * seed) % (p * q);
    return Math.abs(seed);
}
\end{lstlisting}
Generator ten jest nie jest najszybszy, jednakże jest bardzo bezpieczny. Przy odpowiednich założeniach, odróżnienie jego wyników od szumu jest równie trudne co faktoryzacja $N$.\\
\includegraphics[width=\linewidth]{img/bbs-1.png}
%----------------------------------------------------------------------------------------
\subsection{Liniowy Generator Kongruentny (LCG)}
Liniowe Generatory Kongruentne to najbardziej rozpowszechnione w praktyce algorytmy PRNG. Wyróżniamy dwa rodzaje tych generatorów:
\begin{enumerate}
\item Addytywne\\
W postaci: $x_{i+1}=(a*x_{i}+c) \bmod m$
\item Multiplikatywne\\
W postaci: $x_{i+1} = (a * x_{i}) \bmod m$, gdzie:
\end{enumerate}
$x_{i+1}$ to kolejna liczba pseudolosowa,\\
$m, 0 < m$ - zakres generowanej liczby\\
$a, 0 < a < m$ - współczynnik mnożenia\\
$c, 0 \le c < m$ - współczynnik inkrementacji\\
Dla pewnych kombinacji parametrów generowany ciąg jest prawie losowy, dla innych
bardzo szybko staje się okresowy. Cechą tych generatorów jest również fakt, iż jeśli kolejna wygenerowana liczba powtarza się w ciągu liczb już wygenerowanych, to wpadamy w cykl: $x_{1}, x_{2},\dots,x_{1},x_{2}$. Mimo bardzo prostej budowy i zasady działania, powstało wiele prac i analiz matematycznych, które mówią nam o najlepszym doborze parametrów $m, a, c$. Najlepszym, to znaczy takim w którym cykl będzie jak najdłuższy. Przykładowe zalecenia: \\
\begin{enumerate}
\item Parametr $m$ powinien być duży, najlepiej wielokrotnością $2$ (najczęsciej spotyka się generatory LCG, w których $m = 2^{32}$ lub (w nowszych) $m = 2^{64}$)
\item Parametr $a$ powinien być o jeden rząd mniejszy niż $m$.
\item Dobrą wartością dla $a$ jest liczba postaci $t21$, gdzie $t$ - liczba parzysta.
\end{enumerate}

\begin{lstlisting}[style=mystyle, language=java, frame=single, caption = Generowanie następnej liczby pseudolosowej przez LCG]
private final static long a = 25173;
private final static long b = 13849;
private final static long m = 32768;

public static long getRandomNumber() {
    seed = (a * seed + b) % m;
    return seed;
}
\end{lstlisting}
\includegraphics[width=\linewidth]{img/lcg-1.png}
%----------------------------------------------------------------------------------------
\subsection{Mersenne twister}
\textit{Mersenne twister} został opracowany przez Makoto Matsumoto i Takuji Nishimura w 1997 roku\footnote{http://doi.acm.org/10.1145/272991.272995}. Generator ten dostarcza wysokiej jakości liczby pseudolosowe oraz jest bardo szybki. Nazwa pochodzi od tego, że na dlugość okresu została wybrana pierwsza liczba Mersenne'a. Algorytm, mimo swoich zalet, nie nadaje się do zastosowań kryptograficznych. Stosunkowo obserwacja niewielkiej liczby iteracji (624) pozwala przewidzieć wszystkie kolejne. Kolejną kwestią jest dlugi czas inicjalizacji algorytmu - w porównaniu do generatora Fibonacciego lub liniowego generatora kongruencyjnego. 
\includegraphics[width=\linewidth]{img/mt-1.png}
%----------------------------------------------------------------------------------------
\subsection{Generator Park–Miller}
Generator jest odmianą multiplikatywnego liniowego generatora kongruencyjnego, który określamy wzorem:
\[ x_{n+1} = (a * x_{n}) \bmod M \]
gdzie, $x_{n+1}$ to następna liczba pseudolosowa, $a$ - współczynnik generujący kolejną liczbę, $M$ - współczynnik określający zakres generowanych liczb (od $0$ do $M-1$).
\begin{lstlisting}[style=mystyle, language=java, frame=single, caption = Generowanie następnej liczby pseudolosowej przez algorytm Parka-Millera.]
private static final long max = ((long) 2 << 30) - 1;
private static final long a = 16807;

public static long getRandomNumber() {
    seed = (a * seed) % max;
    return seed;
}
\end{lstlisting}
\includegraphics[width=\linewidth]{img/pm-1.png}
%----------------------------------------------------------------------------------------
\subsection{Xorshift}
Algorytm został zaproponowany przez George Marsaglia\footnote{http://www.jstatsoft.org/v08/i14/paper} w 2003 roku. Liczba $x_{n+1}$ jest generowana poprzez wielokrotną różnicę symetryczną liczb $x_{n}$ i przesuniętej bitowo $x_{n}$. Wykorzystanie funkcji $XOR$ sprawia, że ten algorytm jest niezwykle szybki na współczesnych komputerach.
\\
Algorytm ten jest szybki ale nie niezawodny i z pewnością nie nadaje się do zastosowań kryptograficznych. Jednak, połączenie go z nieliniowym generatorem - jak pierwotnie sugerował autor - prowadzi do jednego z najszybszych generatorów, spełniających silne wymogi testów statystycznych.

\begin{lstlisting}[style=mystyle, language=java, frame=single, caption = Generowanie następnej liczby pseudolosowej przez Xorshift]
public static long getRandomNumber() {
    seed ^= seed >> 12;
    seed ^= seed << 25;
    seed ^= seed >> 27;
    seed = (seed * 2685821657736338717L) % max;
    return Math.abs(seed);
}
\end{lstlisting}
Silnik przeglądarki internetowej Webkit korzysta z tego algorytmu przy wywoływaniu $Math.random()$ w języku JavaScript.\\
\includegraphics[width=\linewidth]{img/xorshift-1.png}
%----------------------------------------------------------------------------------------
\section*{Wnioski}

%----------------------------------------------------------------------------------------
%	BIBLIOGRAPHY
%----------------------------------------------------------------------------------------

%\bibliographystyle{unsrt}

%\bibliography{sample}

%----------------------------------------------------------------------------------------

\end{document}